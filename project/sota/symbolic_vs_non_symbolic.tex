\subsubsection{Symbolic vs non-symbolic}
In the last decade, the use of Deep Neural Networks (DNN) gained popularity in any machine learning context with very promising results. However, the need for better soundness and trustworthiness, as well as a better understanding of neural models, led the research community to investigate new methodologies to bring these aspects into the deep learning world. Several new methods to integrate knowledge and DNN have emerged in recent years with this ambition. This line of research is called \textit{Neuro-Symbolic Integration}~(NSI) and it represents a hot topic in the AI community~\cite{3rdWave}. The name combines the terms \textit{neural} and \textit{symbolic}, which respectively refer to neural networks and symbolic reasoning. More in detail, this research area aims to take benefits from both these two subfields of AI to support each other and overcome their limitations.

Starting from DNNs, it is well known they need a big amount of labeled training examples to generalize well. Unfortunately, such quantities of annotated data are not always available due to the big effort needed to be produced. Another limitation is that they lack interpretability because the learned functions are seen as not explainable black-boxes by humans. Furthermore, an issue that may occur when dealing with DNNs is the distributional shift (i.e., the different distribution between the training examples and the real-world data). Moving on to symbolic approaches, they do not suffer from the previous problems. Indeed, they can easily generalize through logical rules without the need for large data. Moreover, they are explainable thanks to the high-level interpretability of the logical knowledge provided by a domain expert. However, concerning DNNs, these kinds of approaches are heavily influenced by noise and are not as efficient as the neural ones.

Depending on the task for which it is designed, an NSI system can combine the symbolic and non-symbolic elements in several ways as we will see in the next section.